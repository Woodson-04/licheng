\documentclass{article}
\usepackage[UTF8]{ctex} 
\usepackage{indentfirst}
\usepackage{array}
\title{REREPORT}
\author{李城}
\date{\today}

\begin{document}

\maketitle

\section{必要函数及实现细节}
\subsection{heapsort.h}
\subsubsection{adjust\_heap()}
递归调整堆:比较某节点和它的两个子节点的大小,将最大的放到当前节点位置,要是确实发生了交换操作,就要递归交换的子节点位置,进行adjust\_heap()操作
\subsubsection{build\_heap()}
设置从大到小遍历依次调整,逐步建立最大堆
\subsubsection{Heapsort()}
从大到小遍历:将最大元素与最后元素(第i个)交换,再对前面i-1个元素调用adjust\_heap(),保持最大堆结构
\subsection{test.cpp}
\subsubsection{四个序列的生成}
全都是1000000大小:orderly顺序填充;random使用mt19937;reverse倒序填充;repeat依靠dis(0,9999)函数填充(0,9999)随机数字
\subsubsection{check()}
依次比较各个位置,当且仅当全一样输出1
\section{测试流程}
依次对random,orderly,reverse,repeat测试我的Heapsort和std::sort\_heap()
\section{列表比较}
\begin{tabular}{|c|c|c|}
\hline
 & my heapsort time & std::sort\_heap time \\
\hline
random sequence & 235.339ms & 158.844ms \\
\hline
ordered sequence & 97.696ms & 66.1131ms \\
\hline
reverse sequence & 100.705ms & 80.5686ms \\
\hline
repetitive sequence & 215.726ms & 164.578ms \\
\hline
\end{tabular}

\end{document}