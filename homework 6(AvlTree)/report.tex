\documentclass{article}
\usepackage[UTF8]{ctex} 
\usepackage{indentfirst}

\title{REREPORT}
\author{李城}
\date{\today}

\begin{document}

\maketitle

\section{remove 函数的实现}

我的 \texttt{remove} 函数主要靠引入父指针实现,这样可以实现 \texttt{remove} 后对各元素的 \texttt{balance}。靠父指针可以从删除位置开始向上检查(我先正常 \texttt{remove},然后反复调用 \texttt{balance} 函数,对可能会出现的不平衡节点进行检查)。
\section{其余内容}
BinaryNode 结构的改动:我改动了 \texttt{BinaryNode} 的结构,添加了父指针和 \texttt{height} 属性。

insert 函数和旋转函数的更新:我改动了 \texttt{insert} 函数、旋转函数,更新了父指针。

我的旋转函数、\texttt{balance} 函数主要参考了课本 \texttt{AvlNode.cpp}。

\end{document}