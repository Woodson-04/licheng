\documentclass{article}
\usepackage{xeCJK}

\title{第四次作业报告}
\author{李城}
\date{October 2024}

\begin{document}

\section*{测试设计}  
你是如何设计\texttt{List.cpp}里的测试的?所有我测试的包括:  
\begin{itemize}  
    \item \texttt{empty()}  
    \item \texttt{size()}  
    \item \texttt{push\_back()}  
    \item 前向迭代器  
    \item \texttt{push\_front()}  
    \item 后向迭代器  
    \item \texttt{front()}  
    \item \texttt{back()}  
    \item \texttt{pop\_front()}  
    \item \texttt{pop\_back()}  
    \item \texttt{erase()}  
    \item \texttt{insert()}  
    \item 范围\texttt{erase()}  
    \item 测试拷贝构造函数  
    \item 测试移动构造函数  
    \item 测试赋值操作符  
    \item 测试移动赋值操作符  
    \item 测试初始化列表构造函数  
    \item \texttt{clear()}  
\end{itemize}  
  
\section*{测试结果}  
\textbf{你的测试结果是什么?}如下:  
\begin{itemize}  
    \item 初始化后列表是否为空: (应为true) \textbf{true}  
    \item 列表大小: (应为0) \textbf{0}  
\end{itemize}  
  
\textbf{调用 \texttt{push\_back(10, 20, 30)} 后}  
\begin{itemize}  
    \item 列表是否为空: (应为false) \textbf{false}  
    \item 列表大小: (应为3) \textbf{3}  
\end{itemize}  
  
\textbf{正向迭代遍历列表元素}  
\begin{itemize}  
    \item (应为10 20 30) \textbf{10 20 30}  
\end{itemize}  
  
\textbf{调用\texttt{push\_front(5)}}  
\begin{itemize}  
    \item 列表大小: (应为4) \textbf{4}  
\end{itemize}  
  
\textbf{反向迭代遍历列表元素}  
\begin{itemize}  
    \item (应为30 20 10 5)  
\end{itemize}  
  
\textbf{第一个元素 (\texttt{front}):} (应为5) \textbf{5}  
  
\textbf{最后一个元素 (\texttt{back}):} (应为30) \textbf{30}  
  
\textbf{调用\texttt{pop\_front()}}  
\begin{itemize}  
    \item 第一个元素 (\texttt{front}): (应为10) \textbf{10}  
    \item 列表大小: (应为3) \textbf{3}  
\end{itemize}  
  
\textbf{调用\texttt{pop\_back()}}  
\begin{itemize}  
    \item 最后一个元素 (\texttt{back}): (应为20) \textbf{20}  
    \item 列表大小: (应为2) \textbf{2}  
\end{itemize}  
  
\textbf{调用\texttt{erase(第二个元素)}}  
\begin{itemize}  
    \item 列表大小: (应为1) \textbf{1}  
    \item 列表元素: (应为10) \textbf{10}  
\end{itemize}  
  
\textbf{在第二个位置插入25}  
\begin{itemize}  
    \item 列表元素: (应为10 25) \textbf{10 25}  
\end{itemize}  
  
\textbf{后面添加元素40、50}  
\begin{itemize}  
    \item 列表元素: (应为10 25 40 50) \textbf{10 25 40 50}  
\end{itemize}  
  
\textbf{调用\texttt{erase(第三个元素到最后一个元素)}}  
\begin{itemize}  
    \item 列表元素: (应为10 25) \textbf{10 25}  
\end{itemize}  
  
\textbf{使用拷贝构造函数创建新列表}  
\begin{itemize}  
    \item 新列表元素: (10 25) \textbf{10 25}  
\end{itemize}  
  
\textbf{使用移动构造函数创建新列表后,原列表为空吗?} (应为false) \textbf{false}  
  
\textbf{移动后的列表元素:} (应为10 25) \textbf{10 25}  
  
\textbf{使用赋值操作符将复制的列表赋值回原列表}  
\begin{itemize}  
    \item 原列表元素: (应为10 25) \textbf{10 25}  
\end{itemize}  
  
\textbf{使用移动赋值操作符后,移动源列表为空吗} (应为true) \textbf{true}  
  
\textbf{移动后的列表元素:} (应为10 25) \textbf{10 25}  
  
\textbf{使用初始化列表创建列表}  
\begin{itemize}  
    \item 列表元素: (应为100 200 300) \textbf{100 200 300}  
\end{itemize}  
  
\textbf{调用 \texttt{clear()} 后,列表为空吗?} (应为true) \textbf{true}  
  
\end{document}  

